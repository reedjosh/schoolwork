
\documentclass[a4paper]{article}

\usepackage{fullpage} % Package to use full page
\usepackage{parskip} % Package to tweak paragraph skipping
\usepackage{tikz} % Package for drawing
\usepackage{amsmath}
\usepackage{hyperref}

\title{Biology 101 HW 3}
\author{Joshua Reed}
\date{\today}

\begin{document}

\maketitle

\section{Draw Reactions}

Draw a simple diagram that shows the inputs and out puts of the light dependent and 
light independent reactions of photosynthesis. Your diagram should show the following: 
What forms of energy and inorganic molecules go into the light dependent reactions, what is 
created by the light dependent reactions and where does it go? What goes into the light 
independent reactions and what are the products? You do NOT need to show the details of the Calvin 
cycle.

\newpage 

\section{Reactions in Photosynthesis} 
Create an outline of the reactions that occur in photosynthesis. Your outline should look like this: 


\begin{enumerate} 
  \item \textbf{Light Dependent Reactions}
  \begin{itemize}
    \item Where they occur
    \begin{itemize}
      \item Photosystem II 
        \begin{itemize}
            \item What does it accomplish
            \item Brief description of how it does it 
        \end{itemize}
      \item Photosystem I 
        \begin{itemize}
            \item What does it accomplish
            \item Brief description of how it does it 
        \end{itemize}
    \end{itemize}
  \end{itemize}
    
  \item \textbf{Light Independent Reactions}
    \begin{itemize}
      \item Where they occur
        \begin{itemize}
          \item Calvin Benson Cycle
          \begin{itemize}
            \item What does it accomplish
            \item Brief description of how it does it 
          \end{itemize}
        \end{itemize}
    \end{itemize}
\end{enumerate}

\newpage

Section{Cell Respiration} 
Draw a simple diagram that shows the inputs and out puts of cell respiration. 
Your diagram should show the following: What molecule enters glycolysis and what molecule 
is passed from glycolysis to the Kreb’s cycle, what products of the Kreb’s cycle are used 
to fuel chemiosmosis; and how many ATP and electron carriers are produced by each glycolysis, 
the Kreb’s cycle and chemiosmosis. You do NOT need to show the details of these stages, 
just what goes in, what comes out, and what is passed from stage to stage. 
\newpage

\section{Electron Transport Chain}
What is an electron transport chain and why is it useful? (In your own words)

\end{document}








