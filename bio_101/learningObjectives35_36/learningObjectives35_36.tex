\documentclass[a4paper]{article}
\raggedright
\usepackage{chemfig}
\usepackage{fullpage} % Package to use full page
\usepackage{parskip} % Package to tweak paragraph skipping
\usepackage{enumerate}

\setcounter{section}{+34}

\title{Learning Objectives 35 \& 36}
\author{Joshua Reed}
\date{\today}

\begin{document}

\maketitle

%%%%%%%%%%%%%%%%%%%%%%%%%%%%%%
% Chapter 35
%%%%%%%%%%%%%%%%%%%%%%%%%%%%%%

\section{Ecology}
\subsection{What Is Ecology?}
\subsubsection{Describe the six levels of organization of organisms, and discuss how environmental challenges impact the range of organisms.
Populations}
\begin{enumerate}
\item Populations: Individuals of the same species that together.

\item Species: All the populations of a particular organism.

\item Communities: Populations of various species that live in the same space.

\item Ecosystems: A community and non-living factors that interact.

\item Biomes: Large areas of plants, animals, and microorganisms that occur together.

\item The Biosphere: All of the worlds biomes.
\end{enumerate}

\subsection{Population Range}
\subsubsection{Identify and describe five key characteristics of populations.}
\begin{enumerate}
  \item Population Range: The area wherin a population occurs.
  \item Population Distribution: The pattern of spacing of individuals.
  \item Population Size: Number of individuals a population contains.
  \item Population Density: Number of individuals per area.
  \item Population Growth: Rate of growth or shrinkage of the population.
\end{enumerate}

\subsection{Population Distribution}
\subsubsection{Describe three ways in which individuals can be distributed.}
\begin{enumerate}
  \item Randomly Spaced: Individuals are randomly spaced, and do not interact strongly with one another. Rare in nature.
  \item Uniformly Spaced: Individuals are uniformly spaced. This often comes about from competition for resources.
  \item Clumped Spacing: Individuals are clumped together. This is often because of micro-habitats and or social structures.
\end{enumerate}

\subsection{Population Growth}
\subsubsection{Contrast the intrinsic versus the actual rate of population increase and exponential versus logistic growth curves.}
The intrinsic growth rate is the growth rate without limits placed upon it: $growth\ rate=r_iN$

The actual accounts for deaths and migration: $r=(b-d)+(i-e)$

\subsection{The Influence of Population Density}
\subsubsection{Differentiate between density-dependent and density-independent effects on population growth.}
Density Dependent Effects are dependent upon the population and act to regulate the populations growth. Density
independent factors such as weather act to regulate growth as well, but are independent of the size of the population.

\subsection{Life History Adaptations}
\subsubsection{Contrast r- and K-selected adaptations.}
r-selected adaptations are adaptations for quick growth--sort of like rabbits with young births of large numbers. k-selected adaptations
are those that help an organism to survive in resource scarce times. 

\subsection{Population Demography}
\subsubsection{Explain how the growth rate of a population is influenced by its age structure, fecundity, and mortality.
How Competition Shapes Communities}
The fecundity is the rate of birth of a cohort. The mortality is the death rate. The difference is the death rate for that cohort or
age group.

\subsection{Communities}
\subsubsection{Contrast individualistic and holistic concepts of community.}
The individualistic concept of community holds that a community is merely a group of species, whereas the holistic concept holds
that the community is something more due to emmergent properties.

\subsection{The Niche and Competition}
\subsubsection{Contrast the fundamental and the realized niche.}
The fundamental niche is the niche an organizm could occupy without competitino and the realized is the niche the organism
occupies in the precense of competition.

\subsubsection{Explain why niche overlap may lead to character displacement.}
When two similar species occupy the same niche, they tend to divert more from their own natural niche to avoid the competition
that would naturaly exist.

\subsection{Coevolution and Symbiosis}
\subsubsection{Describe the three major kinds of symbiotic relationships.}
Mutualism: a symbiotic relationship between organisms that is beneficial to both.

Parasitism: a symbiotic relationship that could be considered predator-prey, and is beneficial to one but harmful to the other.

Commensalism: a symbiotic relationship that is beneficial to one and neither good nor bad for the other.

\subsection{Predator-Prey Interactions}
\subsubsection{Discuss the ways predators can affect prey populations.}
Predators can regulate a poplulation, and cause a predator-prey cycle.

\subsection{Mimicry}
\subsubsection{Contrast Batesian and Müllerian mimicry.}
Batesian Mimicry occurs when an edible species mimics an inedible prey.

Mullerian Mimicry occurs when a species that protects itself is mimiced by a species that doesn't.


\section{The Energy in Ecosystems}
\subsection{Energy Flows Through Ecosystems}
\subsubsection{Distinguish among community, habitat, and ecosystem, and between autotroph and heterotroph.}
Community--all the animals, plants, fungi, and microorganisms that live together.

Habitat--the place that the community lives.

Ecosystem--The sum of the Habitat and the Community.

Autotroph--Those that get their energy from the sun-typically plants.
Heterotroph--Those that get their energy from other organisms.

\subsubsection{Trace the path of energy through the trophic levels of an ecosystem.}
Trophic 1: Plants get energy from the sun.

Trophic 2: Primary Consumers of plants.

Trophic 3: Secondary Consumers that eat the primary.

Trophic 4: Tertiary Consumers that eath the secondary.

Finally, decomposers that turn dead mass into dirt. Bacteria and Fungi.

\subsubsection{Define primary productivity and explain how it is measured.}
Primary productivity is the total amount of energy gained from photosynthetic organisms. It is measured in calories gained per
calories of potential energy.

\subsubsection{Explain why wetlands and rain forests have different net primary productivities.}
Wetlands may gain more energy from sunlight due to algae that doesn't have to waste as much energy building structure as plants.
Biomass to energy is much lower in the wetlands.

\subsection{Ecological Pyramids}
\subsubsection{Explain why a population's pyramid of numbers may not resemble its pyramids of biomass and energy.}
In some cases, the trophic level below another can be consumed so quickly that the levels are inverted. Still, the energy present
at the upper trophic level is less than that of the lower level.

\subsection{The Water Cycle}
\subsubsection{Contrast the environmental and organismic water cycles.}
The environmental water cycle works through evaporation.

The organismic cycle cycles water through the stomata of plants and transpiration.

\subsection{The Carbon Cycle}
\subsubsection{Contrast the effects of respiration, erosion, and combustion on the water cycle.}
Respiration is the creation of \chemfig{CO_2} in the breakdown of food for energy.

Erosion is the natural cycle of \chemfig{CO_2} in the ocean where crustacieans use carbon in the building 
of their shells.

Combustion is the burning of material releasing carbon.

\subsection{Soil Nutrients and Other Chemical Cycles}
\subsubsection{Compare the nitrogen and phosphorus cycles.}
Sulfur is a gas and cycles much more readily than phosphorus. Phosphorus is often found in dirt, and returns to dirt when organisms
decompose.

\subsection{The Sun and Atmospheric Circulation}
\subsubsection{Explain why all the earth's great deserts lie near 30°N or 30°S.}
The earths rotation and inclination causes these parts of the earth to get more sun year round than other places.

\subsection{Latitude and Elevation}
\subsubsection{Describe a rain shadow and explain its cause.}
A rain shadow is a place that is particularly arid due to the moisture absorption capability of the air. This is caused because the 
clouds of rain rise at the mountain, and cause it all to precipitate.

\subsubsection{Explain why changes in latitude and elevation often have similar effects on ecosystems.}
These have similar effects because they both relate to how much sun is recieved.

\subsection{Patterns of Circulation in the Ocean}
\subsubsection{Explain how patterns of oceanic circulation are created and how they affect adjacent lands.}
Oceanic circulation is created from winds and solar energy. This affects adjacent lands by distributing various nutrients such as phosphorus.


\subsubsection{Describe El Niño and explain its cause.}
El Nino is a shifting of weather due to ocean and wind cycling. This occurs every two to seven years.

\subsection{Ocean Ecosystems}
\subsubsection{Compare the marine communities that occur in shallow water, open-sea surfaces, and deep-sea waters.}

\subsection{Freshwater Ecosystems}
\subsubsection{Contrast oligotrophic and eutrophic lakes.}
Oligotropic Lakes have scarce organic matter and nutrients. These lakes are more succeptible to pollution.

Eutrophic lakes are the opposite.

\subsubsection{Differentiate the littoral, limnetic, and profundal zones of a lake.}
Littoral: close to shore.

Limnetic: offshore and open to light.

Profundal: underwater and less affected by light.

\subsubsection{Explain the cause of the spring and fall overturns that occurs in large lakes.}
Stratification causes sprint and fall overturns. This is due to the density of water at different temperatures.

\subsection{Land Ecosystems}
\subsubsection{Identify 10 terrestrial biomes and briefly describe each of them.}
\begin{enumerate}
  \item Rain Forests: Lush and dense. Supports most of earths species.
  \item Savanas: Dry Tropical Grasslands are open and seasonal.
  \item Deserts: Burning hot and sandy. Dry with low vegitation.
  \item Grasslands: Temperate, fertile lands.
  \item Deciduous Forests: Rich hardwood forest with mild climates.
  \item Taiga: Long cold winters, with little rain.
  \item Turndra: Home to large grazing animals, cold, and open.
  \item Chaparral: Evergreens, spiny shrubs, and low trees. Dry summer.
  \item Polar Ice Caps: Cold, no precipitation, and windy. 
  \item Tropical Monsoon Forest: Occurin higher lattitudes, mostly deciduos trees.
\end{enumerate}


\end{document}


