\documentclass[a4paper]{article}
\usepackage{chemfig}
\usepackage{fullpage} % Package to use full page
\usepackage{parskip} % Package to tweak paragraph skipping
\usepackage{enumerate}
\raggedright{}

\setcounter{section}{+37}

\title{Learning Objectives 38}
\author{Joshua Reed}
\date{\today}

\begin{document}

\maketitle

%%%%%%%%%%%%%%%%%%%%%%%%%%%%%%
% Chapter 35
%%%%%%%%%%%%%%%%%%%%%%%%%%%%%%

\section{Human Influences on the Living World}
\subsection{Pollution}
\subsubsection{Explain how biological magnification endangered bald eagles.}
Biological magnification travels through the food chain with increasing potency. By the time these chemicals got to bald eagles, they
were in high enough concentrations to cause significant negative effects. For bald eagles, their eggs were laid with extremely weak shells,
causing them to nearly become extinct.

\subsection{Acid Precipitation}
\subsubsection{Discuss the sources and consequences of acid precipitation.}
Sulfites from the smokestaks of burning coal are largely to blame for acid precipitation. As the sulfites combine with water in the 
air, the water becomes acidic. Rain is typically a PH of 5.6, but it has been recorded as low as 3.0. The consequences of acid
rain include the death of near entire ecosystems. For example, lakes are now nearly barren as the acidic water prevents many fish
from reproducing.

\subsection{Global Warming}
\subsubsection{Assess the argument that global warming is the consequence of increased carbon dioxide in the atmosphere.}
The book says that carbon increase and temperature increase are correlated, and that there is overwhelming consensus among 
scientists. 

\subsubsection{Describe three serious potential impacts of global warming.}
Three potentially serious impacts include effects on rain patterns, agriculture, and sea levels as evidence.

\subsection{The Ozone Hole}
\subsubsection{Explain how chemical coolants caused the ozone hole over Antarctica.}
Ozone acts as a catalyst in the atmosphere to convert protective \chemfig{O_3\ to\ O_2}.

\subsection{Loss of Biodiversity}
\subsubsection{Discuss the impact of three factors thought to play key roles in many extinctions.}
\begin{itemize}
    \item Habitat loss: Through destruction, pollution, disruption, and division.
    \item Species Overexploitation: Even large populations are vulnerable to extinction due to human hunting. Recent historic examples include bison, passenger pigeons, and some whale species.
    \item Introduced Species: Occasionally, a species with be introduced that has no native predators to keep them in check. As such, they tend to crowd out native species.
\end{itemize}

\subsection{Reducing Pollution}
\subsubsection{Describe how economists estimate the “optimal” amount of pollution.}


\subsection{Preserving Nonreplaceable Resources}
\subsubsection{Evaluate the importance of three nonreplaceable resources.}
Economists compare the marginal cost of pollution with the marginal cost of of pollution abatement. Unfortunately, this doesn't take into account the negative
externalities not accounted for in the cost of production that must be paid by those must suffer from pollution.

\subsection{Curbing Population Growth}
\subsubsection{Describe the growth of the human population over the last 10,000 years.}
Human population has been mostly steady for thousands of years, but over the last 300, we've gone from far less than a billion to just over 7 billion.

\subsubsection{State by what percentage the world's human population grows each year, and what has changed to produce this rate of growth.}
The growth rate has been between 2\% and currently 1.2\% over the last hundred years. The industrial revolution and medical advances have caused this growth.

\subsubsection{Explain why population pyramids with broader bases indicate more rapid future population growth.}
The large base of a population pyramid indicates a generation that has not yet had children. Once this generation reaches child bearing age, they will likely
cause rapid growth.

\subsection{Preserving Endangered Species}
\subsubsection{Explain the importance of keystone species to biodiversity.}
A keystone species exerts a strong influence on the structure and function of an ecosystem. Without this species, an entire food chain could collapse.

\subsection{Finding Cleaner Sources of Energy}
\subsubsection{Discuss the potential of biomass as a source of energy.}
Biomass in conjunction with yeasts that can process them have great potential as a source of energy. Many parts of moder agriculture are simply wasted. 
Because fuel made from biomass releases carbon that was only recently pulled from the atmosphere, this cycle produces no net \chemfig{CO_2}.

\subsection{Individuals Can Make the Difference}
\subsubsection{Recount how Lake Washington and the Nashua River were restored through individual action.}
Marion Stoddart organized a cleanup committee and broaght attention to local poleticians. By campaigning, she aided in the passing of the 1966 Clean Water Act, banning dumping.

\end{document}


