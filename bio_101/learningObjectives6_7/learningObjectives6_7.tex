\documentclass[a4paper]{article}
\raggedright
\usepackage{chemfig}
\usepackage{fullpage} % Package to use full page
\usepackage{parskip} % Package to tweak paragraph skipping
\usepackage{enumerate}

\setcounter{section}{+5}

\title{Learning Objectives 6 \& 7}
\author{Joshua Reed}
\date{\today}

\begin{document}

\maketitle

\section{Photosynthesis: Acquiring Energy from the Sun}
\subsection{An Overview of Photosythesis}
\subsubsection{Name the three layers of a leaf through which light must pass.}
The three layers light must pass through are the cuticle, the epidermis, and the mesophyll
cells.

\subsubsection{Provide a brief overview of the photosynthetic events that occur within the
chloroplast.} 
Light penetrates the Thylakoid Surface, absorbing non-green light. Light strikes the photosystem.
Energy is absorbed by chlorophyll pigments. The photosystem is then excited by the energy 
absorbed. Eventually the energy is captured by a special chlorophyll molecule. Then light 
dependent reactions take place making ATP and NADPH. Finally, this reaction facillitates 
non-light dependent reations.

\subsection{How Plants Capture Energy from Sunlight}
\subsubsection{1. Describe what a photon is made of, and state in what way its 
energy is related to its wavelength.}
A photon is a tiny packet of energy. It's energy is related to its wavelength such that
the shorter the wavelength, the greater the energy content.


\subsubsection{Identify what color(s) of light are \it{not} absorbed by the pigment
chlorophyll.}
Reds and Blues are absorbed most efficiently by chlorophyll. Green is not absorbed by
chlorophyll.

\subsection{Organizing Pigments into Photosystems}
\subsubsection{List and describe the five stages of the light-dependent reactions.}
\begin{enumerate}
    \item  Light is captured by a chlorophyll molecule, and the energy is passed from one pigment
            to another.
    \item  An electron of the key chlorophyll molecule is excited. This molecule gives an 
            electron to an electron acceptor, and the electron is replaced by the breakdown of water.
    \item  An excited electron is transported along a series of electron-carriers in the membrane.
            As this electron is transferred, the energy within it is siphoned and used to transport
            hydrogen ions.
    \item  ATP is created from ADP through the process of chemiosmosis. 
    \item  NADPH is created after re-energizing the electron previously transported and siphoned.
\end{enumerate}

\subsubsection{Differentiate reaction center chlorophyll molecules from other photosystem
chlorophyll molecules.}
The reaction center chlorophyll passes an excited electron on to the transport system,
whereas others simply collect energy and pass it along to other chlorophyll molecules.

\subsection{How Photosystems Convert Light to Chemical Energy}
\subsubsection{Describe the function of the electron transport system in noncyclic 
photophosphorylation.}
The electron tranpsort system both siphons energy from the electron, and passes the de-energized
electron on to the next stage or photosystem 1.

\subsection{Building New Molecules}
\subsubsection{Explain why the Calvin cycle requires NADPH as well as ATP.}
Because the Calvin cycle is a way of creating organic molecules, both ATP and NADPH are needed.
ATP provides energy, and NADPH acts as a source of hydrogens and energetic electrons.

\subsubsection{Explain why continuously photosynthesizing cells don't run out of ADP
to us in making ATP.}
Cells don't run out of ADP essentially because ATP and NADPH become ADP and NADP+ 
after use and are recycled.
 

\subsection{Photorespiration}
\subsubsection{Distinguish among C\textsubscript{3}, C\textsubscript{4}, and CAM photosynthesis.}
Plants that use C\textsubscript{3} photosynthesis alone don't do well in hot temperatures
do to photorespiration. C\textsubscript{4} plants have a bundle sheath cell that processes
C\textsubscript{4} to avoid the constraints of photorespiration. CAM plants process C\textsubscript{4}
in hot temperatures, and C\textsubscript{3} in cool temperatures.




%%%%%%%%%%%%%%%%%%
% CHAPTER 7
%%%%%%%%%%%%%%%%%%

\section{How Cells Harvest Energy from Food}
\subsection{Where is the Energy in Food}
\subsubsection{Write a chemical equation for the oxidation of glucose.}
\chemfig{C_6H_{12}O_ + 6O_2 \rightarrow 6CO + 6H_2O + energy}

\subsection{Usign Coupled Reactions to Make ATP}
\subsubsection{State how many molecules of ATP are made from a glucose molecule during glycolysis.}
Four ATP molecules are formed from aglucose molecule during glycolysis. Two per pyruvate.

\subsection{Harvesting Electrons from Chemical Bonds}
\subsubsection{Describe the enzyme that remvoes CO\textsubscript{2} from pyruvate and the
metabolic significance of doing so.}
Pyrovate dehydrogenase removes \chemfig{CO_2} from pyruvate. It contains 60 subunits. While doing so, 
a hydrogen and some electrons are removed from pyrovate and given to \chemfig{NAD^+} to form \chemfig{NADH}.
Eventually pyruvate is joined to coenzime A by pyruvate dehodrogenase. At this point, acetyl-CoA can
be used to form ATP, or if the cell has plenty energy already, it can be used to form fats for later use.

\subsubsection{Identify the substrates for the nine-reaction Krebs cycle and the overall products.}
I'm slightly unclear on this. 10 NADH and two \chemfig{FADH_2} molecules act as electron carriers, and ATP is produced.
It would appear that the substrates are Citrate, Isocitrate, $\alpha$-Ketoglutarate, Succinyl-CoA, Succinate, Fumarate, 
Malate, and Oxaloacetate are the substrates for the nine-reaction Krebs cycle.


\subsection{Using the Electrons to Make ATP}
\subsubsection{Describe the journey of an electron through the electron transport chain, and 
identify its final destination.}
The electron enters NADH dehydrogenase which acts as a hydrogen pump using the electrons energy. Then
\chemfig{FADH_2} passes the electron to the third stage, the \chemfig{bc_1} complex, which also acts as a 
hydrogen/proton pump. The electron is then shuttled to Cytochrome oxidase. Which pumps yet another hydrogen.
Finaly the de-energized electron assists in the formation of a water molecule.

\subsubsection{Calculate how many ATP molecules a cell can be harvested from a glucose molecule in the presence of an oxygen and 
in its abscence.}
36 ATP molecules can be harvested in the prescence of oxygen, I'm not sure how many can be harvested without, but it seems to be less.


\subsection{Cells can Metabolize Food Without Oxygen}
\subsubsection{Distinguish between ethanol fermentation and lactic acid fermentation.}
Ethanol fermentation decarboxylizes NADH, and lactic acid fermentation doesn't. 
Both recycle NADH into \chemfig{NAD^+}


\subsection{Glucose is Not the Only Food Molecule}
\subsubsection{Describe how cells garner energy from proteins and from fats.}

\end{document}


