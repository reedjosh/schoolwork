
\documentclass[a4paper]{article}

\usepackage{fullpage} % Package to use full page
\usepackage{parskip} % Package to tweak paragraph skipping
\usepackage{tikz} % Package for drawing
\usepackage{amsmath}
\usepackage{hyperref}

\title{Economics 201 HW 3}
\author{Joshua Reed}
\date{\today}

\begin{document}

\maketitle

\section{Trade Restriction Article}

\subsection{Select and Article and Provide a Link}
\textbf{\large{US Revises Tariffs and Duties on Chinese Solar Imports}}

\textbf{July 9, 2015}

\textbf{By Feifei Shen}


The U.S. revised some taxes on solar products from certain companies in China to help thwart dumping amid a renewable-energy spat between the two nations.

The U.S. Commerce Department reduced the average rates imposed on most Chinese solar imports in the final decision of a review, it said on its website.

Some units of Yingli Green Energy Holding Co., the second-largest solar manufacturer, received the lowest so-called anti-dumping rate, 0.79 percent. The rate for another group of companies including Canadian Solar Inc., JinkoSolar Holding Co. and some other Yingli units was set at 9.67 percent. Other companies will pay 239 percent.

The Commerce Department also set anti-subsidy rates for most companies at 20.94 percent.

In a preliminary review released in January, the agency recommended reducing the combined anti-dumping/anti-subsidy duties on most Chinese solar manufacturers to about 18 percent from 31 percent.

Import duties may slow the growth of the U.S. solar industry, Jigar Shah, president of the Coalition for Affordable Solar Energy, said in an e-mailed statement.

“Economically counterproductive tariffs have artificially made solar panel prices in the U.S. the most expensive in the world,” Shah said. CASE was formed to represent most of the U.S. solar industry against the petition.

\textbf{\large{Link}}

\textbf{https://www.bloomberg.com/news/articles/2015-07-09/u-s-imposes-dumping-duties-on-imports-of-chinese-solar-goods}

\subsection{Classify the Arguments in the Article}


How would you categorize the trade restrictions discussed in the article (National Defense Argument, Infant Industry Argument, Anti dumping Argument, 
Jobs and Income Argument, Declining Industries Argument or something else)? 

\subsubsection{Arguments That Apply}

\textbf{Infant Industry Argument}

\textit{from Wiki}

The infant industry argument is an economic rationale for trade protectionism. The core of the argument is that nascent industries often do 
not have the economies of scale that their older competitors from other countries may have, and thus need to be protected until they can attain similar economies of scale. 

Solar tariffs and anti-subsidy rates were originally set to help get American solar panel producers up and running. The article claims the cost of solar panels in the US is the highest in the world.
This high cost of panels has helped American panel producers to enter the market.

\textbf{Anti Dumping Argument}

\textit{from Investopedia.com}

An anti-dumping duty is a protectionist tariff that a domestic government imposes on foreign imports that it believes are priced below fair market value. 
Dumping is a process where a company exports a product at a price lower than the price it normally charges on its own home market. 
To protect local businesses and markets, many countries impose stiff duties on products they believe are being dumped in their national market. 

Also mentioned in the article is a so-called 'anti-dumping rate.' The tariffs are in place to prevent Chinese manufacturers from dumping their panels at unfairly below market prices.

The article also talks about the reduction of this tax with the goal of reducing US solar panel prices.


\section{Irrational Decision}

\subsection{What wrong decision did I make due to lack of information.}

I recently accepted a temp position with the hopes of networking with the hiring manager. It turned out the job was completely non-technical and solitary.
Had I known this I would not have accepted the placement.


\subsection{Why did you not acquire the information that was necessary to make a rational decision?}

The information I had access to was asymmetric and imperfect. The placement agency didn't really know much about the position, other than it was with a company I wanted to work for.
The hiring manager within the company knew the job was non-technical, and could tell from my application that I wanted a technical position. Due to the asymmetry of information, the
hiring manager was able to convince me to take a temporary job I did not want. 
\newpage

\section{Demand Supply Curve for British Pounds}

Use the data in the table below to answer the following questions about the market for British pounds. (I) Draw the demand and supply curves for pounds, 
and determine the equilibrium exchange rate (dollars per pound). (II) Suppose that the supply of pounds doubles. Draw the new supply curve. 
(III) What is the new equilibrium exchange rate. (IV) What happens to U.S. imports of British goods? (4 points)

\end{document}






