\documentclass{beamer}

\begin{document}
  \begin{frame}
    \frametitle{Temperature and Birth}
    \framesubtitle{Does birth rate correlate with temperature?}
  \end{frame}
  \begin{frame}
    \frametitle{What is a $\zeta$?}
    \framesubtitle{Temperature and Birthrate for over 100 years}
    Each $\zeta$ is a record of temperature for a month, and a 
    population growth percentage nine months after

    Include Plot of an outcome.
  \end{frame}
  \begin{frame}
    \frametitle{The random sequence in my problem.}
    Temperature average in a month can be viewed as a normal distribution with 
    a mean and variance that is different for each month.

    Growth percentage averaged over a month can be seen as a normal distribution 
    as well. Here the mean and variance may be constant. I will have to 
    analize.

  \end{frame}
  \begin{frame}
    \frametitle{Plot of some outcomes.}
    will include plot soon
    
  \end{frame}
  \begin{frame}
    \frametitle{Stationarity of the random sequences}
    I suspect temperature will be cyclostationary and birthrate will be 
    actually stationary. 
    
    I will have to calculate this.
  \end{frame}
  \begin{frame}
    \frametitle{Expected Value of Sequences}
    Calculate using scipy.
    Plot the data for each.
  \end{frame}
  \begin{frame}
    \frametitle{Cross-correlation}
    Perform this such that the outcome for temperature is compared against 
    months that are 9 out from the original.

    Plot this.
  \end{frame}
  \begin{frame}
    \frametitle{Discuss Outcomes VS Reality}
    \framesubtitle{Compute the student T as related to this situation.}
  \end{frame}
\end{document}
