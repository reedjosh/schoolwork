\mysection{Basic Maths}

\myssection{Series and Sequences}
\mysssection{Geometric Sequence}
A series with a constant ration between successive terms.\\
Ex.  $\frac{1}{2}+ \frac{1}{4}+ \frac{1}{8}+ \frac{1}{16}+\ldots$\\
Often defined as using $ar$\\
Ex.  $a+ar+ar^2+ar^3+\ldots$\\
For $r\neq1$, the sum of the first $n$ terms is:\\
$\sum_{k=0}^{n-1}ar^k=a(\frac{1-r^n}{1-r})$\\
And for infinite sequences:\\
$\sum_{k=0}^{\infty}ar^k=\frac{a}{1-r}$, for $|r|<1$\\

\mysssection{Arithmetic Series}
A series with a constant difference between successive terms.\\
Ex.  $2+5+8+11+\ldots$\\
Sum of an arithmetic series with $n$ terms starting with $a_1$ and ending with $a_2$:\\
$\sum=\frac{n(a_1+a_2)}{2}$

\mysssection{Power Series}
A series of the form:\\
$\sum_{n=0}^{\infty}=a_n{(x-c)}^n$\\
Where often $c=0$\\
$\sum_{n=0}^{\infty}=a_n{(x)}^n$\\
The power series allows generalization of multiplication, division, subtraction, and addition between 
like series. It is also possible to integrate or differentiate a power series.


\mysssection{Taylor Series}
The Taylor series of $f(x)$ (a function that is infinetely differentiable at a number a) is the
power series:
$f(a)+\frac{f'(a)}{1!}(x-a)+\frac{f''(a)}{2!}(x-a)+\ldots$
$\sum_{n=0}^\infty\frac{f^{(n)}(a)}{n!}{(x-a)}^n $

\myssection{Logarithms}
$\log_b c=k$\\
$b^k=c$\\
$\ln(xy)=\ln(x)+\ln(y)$\\
$\ln(x/y)=\ln(x)-\ln(y)$\\
$\ln(x^y)=yln(x)$\\
$\ln(e)=1$\\
$\ln(1/x)=-\ln(x)$

\myssection{Integrals}
$\int x^n dx = \frac{1}{n+1}x^{n+1},n\neq-1$\\
$\int \frac{1}{x} dx = \ln|x|$\\
$\int u dv=uv-\int v du$\\
$\int e^{ax}dx=\frac{1}{a}e^{ax}$
$\int xe^{ax}dx=(\frac{x}{a}-\frac{1}{a^2})e^{ax}$
\myssection{Derivatives}

\myssection{Combinatorics}
\mysssection{Permutations}
Number of ways to order $n$ distinct elements:\\
$n! $
\mysssection{$k$-Permutations of $n$}
Ordered arrangements of a k-element subset of an n-set.
$P(n,k)=\frac{n!}{(n-k)!}$
\mysssection{Permutations With Repitition}
For a set S of size k, the number of n-tuples over S is.
$k^n$
\mysssection{Combination}
$\binom{n}{k}=\frac{n!}{k!(n-k)!}$
\mysssection{Binomial Theorem}



