% Joshua Reed
% Fall, 2017
% 
% hw1.tex
% 
% Homework for introduction to probability.

\documentclass[12pt]{article}
\setlength\parindent{0pt}
 
\usepackage[margin=1in]{geometry} 
\usepackage{amsmath,amsthm,amssymb}
\usepackage{pgfplots}


\makeatletter
\renewcommand{\@seccntformat}[1]{}
\makeatother

% For Align:
%'*' tells LaTeX not to number lines.
%Align is a math environment. Thus \text{} is used for text contained within.
%'&' indicates a seperation between columns.

\begin{document}

{%Header section
  \large \bfseries 
  Joshua Reed \\
  Fall, 2017 \\
  \begin{center}
    {\huge Homework 2}\\
    MTH 361 - Introduction to Probability% chktex 8 
  Section 1.7 39, 41, 47, 53, and 55.
  \end{center}}
 
 
\section{1.7 \#39} 
\subsection{Exercise}How many times should a coin be tossed so that the probability of at least one head is $\geq 99\%$?

\subsection{Solution}
\subsubsection{Complement}
The probability of flipping 1 or more heads is the complement of the probabilty of flipping 0 heads.
\begin{align*}
P(X\geq 1) = 1-P(X=0)
\end{align*}

\subsubsection{Applied}
The probablity of flipping 0 heads is the probabilty of flipping n tails in a row.
\begin{align*}
P(X\geq 1) &= 1-P(X=0)\\
0.99 &\leq 1-(\frac{1}{2})^n\\
-0.01 &\leq -(\frac{1}{2})^n\\
0.01 &\leq (\frac{1}{2})^n\\
n &\geq log_\frac{1}{2}(0.01) \\
n &\geq 6.64 \\
n & = 7 
\end{align*}

And as verification.
\begin{align*}
P(X\geq 1) & \stackrel{?}{\geq} 1-P(X=0)\\
0.99& \stackrel{?}{\geq} 1-(\frac{1}{2})^7\\
& \stackrel{?}{\geq} 1-0.0078\\
& \stackrel{\checkmark}{\geq} 0.9922
\end{align*}

Thus 7 trials gurantee at least a $99\%$ chance of flipping at least one heads.

\section{1.7 \#41} 
\subsection{Exercise}
A bet is said to carry 3 to 1 odds if you win \$3 for every \$1 you bet. What must the probability of winning be for this
to be a fair bet?

\subsection{Solution}
\subsubsection{Fair Bet}
A fair bet is taken to mean an expectation of zero.

\subsubsection{Odds}
Setup the expectation equation.
\begin{align*}
E[X] & = \$3P(\alpha)-\$1P(\alpha^c)\\
\$0 & = \$3P(\alpha)-\$1P(\alpha^c)\\
\end{align*}

Account for total probability of 1.
\begin{align*}
1 = P(\alpha)+P(\alpha^c)
\end{align*}

Solve for $P(\alpha^c)$
\begin{align*}
P(\alpha^c) & = 1-P(\alpha)\\
\end{align*}

Insert into the first equation.
\begin{align*}
\$0 &= \$3P(\alpha)-\$1(1-P(\alpha))\\
0 &= 3P(\alpha)-1+1P(\alpha)\\
1 &= 4P(\alpha)\\
P(\alpha) &= \frac{1}{4}\\
\end{align*}

\subsubsection{P(Winning)}
Thus the probabilty of winning must be $\frac{1}{4}$.

\subsubsection{Check}
\begin{align*}
\$0 & \stackrel{?}{=} \$3(\frac{1}{4})-\$1(\frac{3}{4})\\
&\stackrel{\checkmark}{=} \$0.75 - \$0.75
\end{align*}


\section{1.7 \#47} 
\subsection{Exercise}
Five people play a game of `odd man out'' to determine who will pay for the pizza they ordered. Each flips a coin. 
If only one person gets heads (or tails) while the other four get tails (or heads) then he is the odd man and has to pay.
Otherwise, they flip again. What is the expected number of tosses needed to determine who will pay?

\subsection{Solution}
\subsubsection{P(Game End)}
The probability of successfully deciding who pays is the number of ways the game can end divide by the total number
of outcomes. 
\subsubsection{Total Outcomes}
Each coin can be either heads or tails.
\begin{align*}
|\Omega|&=2^n\\
&=2^5\\
&=32\\
\end{align*}
\subsubsection{Number of Game Ending Events}
The nuber of ways the game can end is the number of ways to place either a tails or a heads in a set of their opposites.
\begin{align*}
|E|&=2(\text{Number of Ways to Place in 5})\\
&=2*5\\
&=10\\
\end{align*}
\subsubsection{Game Ending Probability}
Thus the probability of the game ending on any one round is $\frac{10}{32} = \frac{5}{16}$.
\subsubsection{Geometric Distribution}
This is related to the geometric distribution wherin X is the number of trials before a success and p is the 
probabilty of success.
\begin{align*}
P(X=k)=p(1-p)^{k-1}
\end{align*}
\subsubsection{Expected Value of the Geometric Distribution}
\begin{align*}
\mu_x&=\sum_{k=1}^{\infty}(\text{Value k})P(X=k)\\
&=\sum_{k=1}^{\infty}kp(1-p)^{k-1}\\
&=p\sum_{k=1}^{\infty}k(1-p)^{k-1}\\
&=p[1(1-p)^0+2(1-p)^1+3(1-p)^2+...]\\
\mu_x(1-p)&=p[1(1-p)^1+2(1-p)^2+3(1-p)^3+...]\\
\end{align*}
Subtract the above two lines\ldots
\begin{align*}
\mu_x-\mu_x(1-p)&=p[(1-p)^0+(1-p)^1+(1-p)^2+...]\\
\mu_x(1-(1-p))&=p[(1-p)^0+(1-p)^1+(1-p)^2+...]\\
\mu_x&=[(1-p)^0+(1-p)^1+(1-p)^2+...]\\
\end{align*}
Which is an infinite geometric series with $\sum = \frac{1}{1-r}$
\begin{align*}
\mu_x &= \frac{1}{1-(1-p)}\\
&= \frac{1}{p}
\end{align*}

\subsubsection{Expected Number of Rounds}
The probability of a game ending event is $p=\frac{5}{16}$. Therefore the expected number of rounds before a game 
ending event is $E_x=\frac{1}{\frac{5}{16}}=\frac{16}{5}=3.2$ rounds.

\section{1.7 \#47} 
\subsection{Exercise}
The Elm Tree golf course in Corland, NY is a par 70 layout with 3 par fives, 5 par threes, and 10 par fours. Find 
the mean and variance of par on this course.

\subsection{Solution}
\subsubsection{Expectation}
The total number of holes is required to find the expectation.

\begin{align*}
\sum &= 3 + 5 + 10\\
&= 18
\end{align*}
The mean is the total par over the total number of holes.

\begin{align*}
\mu_x &= \frac{70}{18}\\
&= 3.889\end{align*}

\subsubsection{Variance}
\begin{align*}
\sigma_x^2 = \sum(x_i-\mu_x)^2p_i\\
 = (5-3.889)^2\frac{3}{18}+(3-3.889)^2\frac{5}{18}+(4-3.889)^2\frac{10}{18}\\
 = (1.111)^2\frac{3}{18}+(-0.889)^2frac{5}{18}+(0.111)^2\frac{10}{18}\\
 = 0.2056+0.220+0.007\\
 = 0.4326
\end{align*}

\section{1.7 \#55} 
\subsection{Exercise}
Can we have a random variable with $E[X]=3$ and $E[X^2] = 8$?

\subsection{Solution}
\subsubsection{Reasoning}
An expectation of 3 means that all of the values times their probability summed equals 3. 

A second moment of 8 means that those values squared times their respective probabilities summed equals 8.

I don't see why this wouldn't be doable.
\subsubsection{Attempt to Fabricate Example}
\begin{align*}
\mu_x=2(1/3)+2(1/3)+2(1/3)\\
E[X^2]=3(3/4)+1(1/4)\\
E[X^2]=16(3/4)
\end{align*}

After a few attempts it becomes apparent that I don't actually know an easy way to solve this. It may or may not be
possible, but I believe it is as there are so many possible combinations of outcomes and probabilities. I will
have to ask the solution after class.


\end{document}










