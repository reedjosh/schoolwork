% Joshua Reed
% Fall, 2017
% 
% hw3.tex
% 
% Homework for introduction to probability.

\documentclass[12pt]{article}
\setlength\parindent{0pt}
 
\usepackage[margin=1in]{geometry} 
\usepackage{amsmath,amsthm,amssymb}
\usepackage{pgfplots}


\makeatletter
\renewcommand{\@seccntformat}[1]{}
\makeatother

% For Align:
%'*' tells LaTeX not to number lines.
%Align is a math environment. Thus \text{} is used for text contained within.
%'&' indicates a seperation between columns.

\begin{document}

{%Header section
  \large \bfseries 
  Joshua Reed \\
  Fall, 2017 \\
  \begin{center}
    {\huge Homework 2}\\
    MTH 361 - Introduction to Probability \\% chktex 8 
  Section 2.7: 7, 25, 33, 41, 53---Section 3.5: 7, 24, 29, 35, 37
  \end{center}}
 
 
\section{2.7 \#7} 
\subsection{Exercise}A commercial for Glade Plug-ins says that by inserting 2 of a choice of 11 scents into the device, you can make more than 50 combinations. If we exclude 
the boring choice of two of the same scent, how many possibilities are there?

\subsection{Solution}
\subsubsection{Ways to pick the first.}
There are 11 ways to pick the first item. 

\subsubsection{Ways to pick the second.}
There are then 10 ways to pick the second item. 

\subsubsection{Total ordered pickings}
There $11*10$ ways to pick two different scents; however since the order of picking is not relavent, this duplicates the number of combinations by $2$. 

Thus there are \textbf{$55$} ways to combine $2$ different scents from $11$.

\section{2.7 \#25} 
\subsection{Exercise}
How many ways can 5 history books, 3 math books, and 4 novels be arranged on a shelf if the books of each type must be together.


\subsection{Solution}
\subsubsection{Frame the problem.}
Because the books have to be together, this is how many ways can the books be arranged within their categories 
times how many ways can the categories be arranged?

\subsubsection{Category Arrangement}
When viewing the problem as how to place three items, placing each item reduces the number of placement choices
by one.

In placing the first book, there are three choices, then two for the second, and only one remaining for the final
item.

As such, there are $3! $ or $3*2*1=6$ ways to arrange the items.

\subsubsection{Individual Book Arrangements}
This is the number of arrangements within each category times themselves.

\begin{align*}
  \text{Arrangements Within Categories}&=3!4!5! \\
  &= 17,280
\end{align*}

\subsubsection{Total Arrangements}
\begin{align*}
  \text{Total Arrangements }&=17,280*6 \\
  &=103,680
\end{align*}

\section{2.7 \#33} 
\subsection{Exercise}
David claims to be able to distinguish brand B beer from brand H, but Alice claims that he just guesses. 
They set up a taste test with 10 small glasses of beer. David wins if he gets 8 or more right. WHat is the probability
that he will win (a) if he's just guessing? (b) If he gets the right answer with probability 0.9?

\subsection{Solution}
\subsubsection{When David is Just Guessing}
Assuming when David is just guessing, he guesses correct with $P($Correct Beer$)= 0.5$.

In order to win he needs to get 8 or more correct. Thus, this is $P(X\geq8)$.

The probability of $P(X=x)$ is simply the binomial distribution.

This can be solved using $P(X\geq8) = P(X=8 ) +P(X=9)) + P(X=10)$

\begin{align*}
  P(X=k)=\binom{10}{k}p^kq^{n-k}
\end{align*}
\begin{align*}
  P(X\geq8)&=\binom{10}{8}0.5^{10} + \binom{10}{9}0.5^{10} + \binom{10}{10}0.5^{10}\\
  &=9.77*10^{-4}\left(\binom{10}{8} + \binom{10}{9} + \binom{10}{10}\right)\\
  &=0.054688
\end{align*}

\subsubsection{When David Has $P($Correct Beer$)= 0.9$.}
This is the same problem, but with a much greater probability of success. 

Again\ldots 
\begin{align*}
  P(X=k)=\binom{10}{k}p^kq^{n-k}
\end{align*}
and here, $q = 1-p$.

At this point I simply used python to compute 
\begin{align*}
  P(X\geq8)&=1-F_x(7)\\
  &=0.929809
\end{align*}


\section{2.7 \#41} 
\subsection{Exercise}
In one of the New York state lottery games, a number is chosen at random between 0 and 999. Suppose you play this
game 250 times. Use the Poisson approximation to estimate the probabiltiy that you will never win and
compare with the exact answer.

\subsection{Solution}
\subsubsection{Exact}\label{ssub:approximation}
The exact solution is the binomial with $P(X=0)$.
\begin{align*}
  P(X=k)&=\binom{250}{0}\frac{1}{1000}^0\frac{999}{1000}^{250}\\
        &=0.77870338
\end{align*}



\subsubsection{Approximation}\label{ssub:approximation}
Poisson Approximation:
$\lambda = pn=\frac{250}{1000}=0.25$

\begin{align*}
  P(X=0)&=\frac{(\lambda t)^0}{0!}e^{\lambda t}\\
         &=\frac{1}{1}e^{0.25 t}\\
        &=0.0.7788008
\end{align*}

In comparison the two answers are very close. It seems the larger the n, the closer the two outcomes become.


\section{2.7 \#53} 
\subsection{Exercise}
Four people are chosen at random from 5 couples. What is the probability that two men and two women are selected?

\subsection{Solution}
\subsubsection{Frame the Problem}\label{ssub:frame_the_problem}
This is the number of ways to choose 2 men from five, then 2 women from 5 divided by the total
number of ways to choose 4 people from 10.

\begin{align*}
  P(\text{2 and 2})&=\frac{\binom{5}{2}^2}{\binom{10}{4}}\\
  &=\frac{100}{210} =\frac{10}{21}
\end{align*}

\section{3.5 \#7} 
\subsection{Exercise}
Suppose 60\% of the people subscribe to newspaper A, 40\% to newspaper B, and 30\% to both. If a person is picked at
random, what is the probability that she subscribes to newspaper A?

\subsection{Solution}
The solution is in wording the problem as `What is the probability of A given A union B?'

\begin{align*}
  P(A|(A\cup B)&= \frac{P(A\cap (A\cup B))}{P(A\cup B)}\\
\end{align*}

Using the mutual exclusion principle\ldots
\begin{align*}
  P(A\cup B) &=P(A) + P(B) - P(A\cap B)\\
             &=0.6 + 0.4 - 0.3\\
             &=0.7
\end{align*}

And noting that \ldots
\begin{align*}
  P(A\cap(A\cup B))=P(A)
\end{align*}

Finally
\begin{align*}
  P(A|(A\cup B)&= \frac{P(A)}{P(A\cup B)}\\
               &= \frac{0.6}{0.7}
\end{align*}

\section{3.5 \#24} 
\subsection{Exercise}
John takes the bus with probability 0.3 and the subway with probabilty 0.7. 
He is late 40\% of the time he takes the bus, but only 20\% of the time when he takes the subway. What is the
probability that he is late for work?

\subsection{Solution}
This is just a sort of weighted probability sum. 

\begin{align*}
  P(\text{late})&= P(\text{Sub})P(\text{Late}|\text{SUB}) + P(\text{Bus})P(\text{Late}|\text{Bus})\\
                &= 0.3*0.4 + 0.7*0.2\\
                &= 0.12 + 0.14\\
                &= 0.26
\end{align*}

A point of note here is that $P($Sub$)$ and $P($Bus) are probably better thought of as $P(S)$ and $P(\neg S)$.

Then the total probability of being late L becomes\ldots
\begin{align*}
  P(L)=P(\neg S)P(L|\neg S) + P(S)P(L|S)
\end{align*}


\section{3.5 \#29}
\subsection{Exercise}
A student is taking a multiple-choice test in which each question has four possible answers. She knows the answers to 50\% of the questions, can narrow the choices down to two 30\% of the time, and does not know anything about 20\% of the questions. What is the probability that she will correctly answer a question chosen at random from the test?

\subsection{Solution}
\begin{align*}
  P(C) &= P(C|X)P(X) + P(C|Y)P(Y) + P(C|Z)P(Z)\\
       &= 1.0(0.50) + 0.50(0.30) + 0.25(0.20)\\
       &= 0.50 + 0.15 + 0.05\\
       &= 0.70
\end{align*}

\section{3.5 \#35}
\subsection{Exercise}
The alpha fetal protein test is meant to detect spina bifida in unborn babies, a condition that affects 1 out of 1,000 children who are born. The literature on the test indicates that 5\% of the time a healty baby will cause a positive reaction.
We will assume that the test is positive 100\% of the time when spina bifida is present. Your doctor has just told
you that your alpha fetal protein test was positive. What's the chance that your baby actually has
spina bifida?

\subsection{Solution}
\begin{align*}
  P(\text{Pos}|\text{TestPos})&=\frac{P(\text{TestPos}|\text{Pos})P(\text{Pos})}{P(\text{TestPos})}\\
                              &=\frac{(1)P(\text{Pos})}{P(\text{Pos})+P(\text{TestPos}|\text{Neg})}\\
                              &=\frac{0.001}{0.001+0.05}\\
                              &=0.0197
\end{align*}
So, less than 2\% chance that your baby actually has spina bifida.

\section{3.5 \#37}
\subsection{Exercise}
To improve the reliability of the channel described in the last example, we repeat each digit in the message three times.
What is the probability that 111 was sent given that (a) we recieved 101, (b) We received 000?

As per the previous problem, p=0.9, q=0.1. Ones are as likely as zeroes.

\subsection{Solution}
\subsubsection{Received a 101}\label{ssub:received_a_101}

\begin{align*}
  P(111|101) = \frac{P(101|111)P(111)}{P(101)}
\end{align*}

Because the sent digits have to be either 111 or 000, the P(111)=0.5, and P(000)=0.5.

From here, the probability that 111 was sent given 101 was received is\ldots

\begin{align*}
  P(111|101) &= \frac{P(101|111)}{P(101|000) + P(101|111)}\\
             &= \frac{(0.1^1*0.9^2)}{0.1^3 + (0.1^1*0.9^2)}\\
             &= \frac{0.081}{0.001 + 0.081}\\
             &= 0.987805
\end{align*}








\end{document}










