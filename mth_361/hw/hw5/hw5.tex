% Joshua Reed
% Fall, 2017
% 
% hw3.tex
% 
% Homework for introduction to probability.

\documentclass[12pt]{article}
\setlength\parindent{0pt}
 
\usepackage{pgfplots}
\usepackage{pgfplotstable}
\usepackage{array}
\usepackage[margin=1in]{geometry} 
\usepackage{amsmath,amsthm,amssymb}
\usepackage{float}
\usepackage{enumitem}
\usepackage{bm}


\makeatletter
\renewcommand{\@seccntformat}[1]{}
\makeatother

\newcommand{\mkmatrix}[2]{
  \pgfplotstableread{#1}\mytable
  \pgfplotstabletypeset[
    begin table=\begin{bmatrix},
    end table=\end{bmatrix},
    header=false, every head row/.style={output empty row},
    skip coltypes, 
    write to macro=#2,
    typeset=false]{\mytable}}

% For Align:
%'*' tells LaTeX not to number lines.
%Align is a math environment. Thus \text{} is used for text contained within.
%'&' indicates a seperation between columns.

\begin{document}

{%Header section
  \large \bfseries 
  Joshua Reed \\
  Fall, 2017 \\
  \begin{center}
    {\huge Homework 5}\\
    MTH 361 - Introduction to Probability \\% chktex 8 
  \end{center}}
 
 
\section{Exercise 1}
Starting with \$2, you play a game where there is  an equal chance of winning or losing. If you win, 
you gain \$1, and if you lose, you lose \$1. You continue playing until you win \$4, or lose all your money.

This is a version of ``Gambler's Ruin''.

% load matrices created via python
\mkmatrix{p1_data/TM.csv}{\TM}
\mkmatrix{p1_data/TM2.csv}{\TMS}
\mkmatrix{p1_data/ID.csv}{\ID}
\mkmatrix{p1_data/OD.csv}{\OD}


\begin{enumerate}[label=(\alph*)]
  \item Write down a Markov chain describing this situation. 

  The state space is $\{\$0, \$1, \$2, \$3, \$4 \}$
  
  The initial distribution is $\bm{i}=\ID$ as the gambler starts with \$2.
  
  The transition matrix is $\TM$. 

  With sink states in rows $0$ and $4$ representing the fact that 
  once our gambler reaches either \$0 or \$4, he stops gambling.
    

\begin{align*}
  f_x(x|n=2)&=\bm{i}\bm{T}^2\\
            &=\ID\TMS\\
            &=\OD
\end{align*}
  \item The chance of going broke after two rounds is $1/4$.
  \item The chance that the gambler is still playing is $1/2$ in which case he still has \$2.
\end{enumerate}
\newpage









\section{Excercise 2}
There are two habitats that a certain kind of migratory bird likes to occupy. When you start 
studying these birds, 30\% are in habitat A and 70\% are in habitat B. Each year, the birds 
migrate according to a Markov chain with transition matrix:


% Setup matrix macros.
\newcommand{\PITWO}{%
  \begin{bmatrix}
    2/3 & 1/3 \\
  \end{bmatrix}}

\newcommand{\PIONE}{%
  \begin{bmatrix}
    3/4 & 1/4 \\
  \end{bmatrix}}

\renewcommand{\ID}{%
  \begin{bmatrix}
    3/10 & 7/10 \\
  \end{bmatrix}}

\renewcommand{\TM}{%
  \begin{bmatrix}
    7/8 & 1/8 \\
    2/8 & 6/8
  \end{bmatrix}}

\begin{align*}
  \bm{P} = \TM
\end{align*}
\begin{enumerate}[label=(\alph*)]
  \item What fraction of the population is in each habitat after 1 year?
    
    Here the initial distribution $\bm{i}=\ID$.
    \begin{align*}
      f_x(x|n=1)&=\bm{i}\bm{P}^1\\
                &=\ID\TM\\
                &=
        \begin{bmatrix}
          0.4375 & 0.5625
        \end{bmatrix}
    \end{align*}
  
    As such, 43.75\% are in habitat A and 56.25\% are in habitat B.
  \item Which of the following describes the equilibrium distribution for this Markov chain?
  
    $$\pi_1=\PIONE \text{\qquad or \qquad} \pi_2=\PITWO$$

    Here $\pi_2$ describes the equlibrium distribution for this Markov chain. The reason is
    when multiplied by $\bm{P}$ the output remains $\pi_2$.

  \item As $n_{years}\to\infty$ the distribution of birds between the two habitats eventually
    reach the equilibrium distribution of $\pi_2$.
\end{enumerate}
\newpage

\section{Exercise 3}
At a certain college, 63\% of premed students switch to liberal arts before graduation and 18\%
of liberal arts majors switch to premed. Given that 60\% of incoming students are premed when
they matriculate and 40\% are liberal arts majors, what percent of the graduating class will be
premed?


% Setup matricies again.
\renewcommand{\ID}{%
  \begin{bmatrix}
    6/10 & 4/10 \\
  \end{bmatrix}}

\renewcommand{\TM}{%
  \begin{bmatrix}
    0.37 & 0.63 \\
    0.18 & 0.82 \\
  \end{bmatrix}}

Here, the initial distribution $\bm{i}=\ID$.

And the transition matrix $\bm{P}=\TM$

The transition is simply one time step or $n=1$.

\begin{align*}
  f_X(x|n=1)&=\bm{i}\bm{P}^1\\
            &=\ID\TM\\
            &=
    \begin{bmatrix}
      0.294 & 0.706
    \end{bmatrix}
\end{align*}

Here the percent that will graduate premed is 29.4\%.
\newpage

\section{Exercise 4}
Classify the following Markov chains as irreducible, aperiodic, both, or neither. 

\begin{enumerate}[label=(\alph*)]
  \item $$\bm{P}=\begin{bmatrix}0 & 1 \\ 1 & 0\end{bmatrix}$$

    This is both irreducible and periodic. 

  It is irreduccible as both states communicate.

    The state always return to itself with a period of $n=2$

  \item$$\bm{P}=\begin{bmatrix}1 & 0\\1/2 & 1/2\end{bmatrix}$$
    
    This isn't irreducible as it contains two states in which one is a sink. This means
    the two states cannot be viewed as one. 

    This is aperiodic as it can stay in the same state over one transition.

  \item$$\bm{P}=\begin{bmatrix}1/2 & 1/2 & 0\\ 1/2 & 0 & 1/2 \\ 0 & 1/2 & 1/2\end{bmatrix}$$

    This is irreducible as all states communicate. 

    This is aperiodic as it can return to state B in any number of steps greater than 1 and 
    contains states that can return to themselves.


\end{enumerate}
\newpage




\section{Exercise 5}
A given stock behaves according to the following Markov chain. At the end of any day, the stock price may increase, stay the same, or decrease. Call these states I, S, and D respectively. 


\renewcommand{\TM}{%
  \begin{bmatrix}
    2/3 & 1/3 & 0 \\
    1/3 & 1/3 & 1/3 \\
    0   & 1/3 & 2/3
  \end{bmatrix}}
\renewcommand{\ID}{%
  \begin{bmatrix}
    1 & 0 & 0 \\
  \end{bmatrix}}
\mkmatrix{p5_data/TM3.csv}{\TMThree}
\mkmatrix{p5_data/OD.csv}{\OD}

$$\bm{P}=\TM$$


Given that a certain stock started at had increased on the previous day, what is the 
probability that the stock will not decrease 3 days from now?

The input distribution $\bm{i}=\ID$

This is simply $P(x\neq D|n=3) = 1-P(x= D|n=3)$.

where 

\begin{align*}
  f_X(x|n=3)&=\bm i \bm p^3\\
                &=\ID\TMThree\\
                &=\OD
\end{align*}

The above is rounded, so the actual value of $P(x=D|n=3)=0.\overline{ 185}$

Finally $P(x\neq D | n=3)=1-0.\overline{185}=0.\overline{814}$




\end{document}










