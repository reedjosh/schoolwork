\section*{Moments}
\subsection*{First}

\subsubsection*{General}
The first moment is the mean of the distribution. Sometimes refered to as the center of mass.

\subsubsection*{Formula}
Where $p(x)$ is the probabilty of the outcome $x$ occuring.
$\mu_x=E\{X\}=\int p(x)xdx$\\
And applies via a sum for the discrete case.

\subsection*{N\textsuperscript{th} Moment}
$E\{X^n\}=\int p(x)x^ndx$\\


\subsection*{Variance of $X$}    

$\sigma_x^2 = E\{[X-m_x]^2\}=E\{X^2\}-\mu_x^2$

\subsubsection*{Properties}
If $Y=aX+b$, \\ 
then $m_y=am_x+b$ \\
and 
$\sigma_y^2=a^2\sigma_x^2$

\subsection*{Expectations}
\subsubsection*{General}
The expectation $E$ of a function $g$ of a random variable $x$, $E\{g(X)\}$: \\
$E\{g(X)\}=\int_{-\infty}^{\infty}g(u)f_x(u)du$ \\
A sum can be substituted for the integral in the discrete case unless using impulse functions

\subsubsection*{Properties}
$E\{C\}=C$ \\
$E\{ag(X)+bh(X)\}=aE\{g(x)\}+bE\{h(X)\}$ \\
If $g(X)\geq 0$, then $E\{g(X)\} \geq 0$


    

